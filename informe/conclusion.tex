Como conclusión, se pudo implementar un sistema que corra 16 tareas concurrentemente, con funciones únicas a cada tarea, que esté atento a interrupciones, tanto externas como internas. También se pudo comprender cómo funciona el direccionamiento de la memoria. A su vez las tareas se pudieron intercambiar satisfactoriamente. Otros puntos que se pudieron implementar son un sistema básico de $I/O$ en C para la pantalla y una dinámica de juego aceptable. Con este trabajo se pudo entender mejor como funciona un sistema complejo con varias tareas, las funcionalidades y la arquitectura de los procesadores Intel, que entiendo será un tema básico a saber en la próxima materia de Sistemas Operativos. 