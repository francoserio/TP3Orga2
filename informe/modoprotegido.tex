El \textit{Modo Protegido}, es un poco mas completo que el \textit{Modo Real}, tenemos 4GB de memoria disponible, 4 niveles de protecci\'on de memoria y rutinas de atencii\'on con privilegios.

\subsection{Segmentacion}
Nos encargamos de setear los offset de los descriptores de los selectores de segmentos. Luego seteamos la pila del kernel en la direcci\'on 0x27000

\begin{algorithmic}
\State \tab BITS 32
\State \tab mp:

    \State \tab \tab ; Establecer selectores de segmentos
    \State \tab \tab xor eax, eax
    \State \tab \tab  mov ax, 1011000b
    \State \tab \tab  mov ds, ax
    \State \tab \tab  mov ss, ax
    \State \tab \tab  mov es, ax
    \State \tab \tab  mov gs, ax
    \State \tab \tab  mov ax, 1100000b
    \State \tab \tab  mov fs, ax
    \State \tab \tab  ; Establecer la base de la pila
    \State \tab \tab  xchg bx, bx
    \State \tab \tab  mov esp, 0x27000
    \State \tab \tab  mov ebp, 0x27000    
\end{algorithmic}

Luego de esto imprimimos en pantalla un mensaje de bienvenida a nuestro sistema e inicializamos la pantalla.
\begin{algorithmic}
    \State \tab \tab ; Imprimir mensaje de bienvenida
    \State \tab \tab $imprimir_texto_mp iniciando_mp_msg, iniciando_mp_len$, 0x07, 2, 0
    \State \tab \tab call $screen_inicializar$
    \State \tab \tab call $screen_pintar_nombre$
    \State \tab \tab call $screen_pintar_puntajes$
\end{algorithmic}

Las funciones \textit{$screen\_inicializar$}, \textit{$screen\_pintar\_nombre$} estan definidas en \textbf{screen.h} y desarrolladas en \textbf{screen.c}. Estos m\'etodos se encargan de pintar la pantalla, seg\'un lo pedido en el enunciado.

\subsection{Paginado}
Habilitamos la paginaci\'on en nuestro sistema de la siguiente manera:
\begin{algorithmic}
    \State \tab ; Cargar directorio de paginas
    \State \tab ; Inicializar el directorio de paginas
    \State \tab call $mmu_inicializar_dir_kernel$
    \State \tab ; Habilitar paginacion
    \State \tab xor eax, eax
    \State \tab mov eax, cr0
    \State \tab or eax, 0x80000000
    \State \tab mov cr0, eax
\end{algorithmic}

Para mappear y unmapear definimos m\'etodos en el archivo \textit{mmu.c}.
Para mappear:
\begin{algorithmic}
    \State \tab void $mmu_mapear_pagina$(unsigned int virtual, unsigned int cr3, \State \tab unsigned int fisica, unsigned char $read_write$, unsigned char $user_supervisor$) {

  \State \tab \tab $page_directory_entry$* pd = ($page_directory_entry$*)(cr3);
  \State \tab \tab unsigned int indiceDirectory = virtual $>>$ 22;
  \State \tab \tab unsigned int indiceTable = (virtual $>>$ 12) $<<$ 10;
  \State \tab \tab $page_directory_entry$ pde = pd[indiceDirectory];
  \State \tab \tab if (pde.present == 1) {
    \State \tab \tab \tab $page_table_entry$* pt = ($page_table_entry$*)((pde.$base_address$ << 12) >> 12);
    \State \tab \tab \tab $page_table_entry$ pte = pt[indiceTable];
    \State \tab \tab \tab if (pte.present == 1) {
      \State \tab \tab \tab \tab $pte.user_supervisor$ = $user_supervisor$;
      \State \tab \tab \tab \tab $pte.read_write$ = $read_write$;
      \State \tab \tab \tab \tab $pte.base_address$ = (fisica $>>$ 12);
    \State \tab \tab \tab } else {
      \State \tab \tab \tab \tab pte.present = 1;
      \State \tab \tab \tab \tab $pte.user_supervisor$ = $user_supervisor$;
      \State \tab \tab \tab \tab $pte.read_write$ = $read_write$;
      \State \tab \tab \tab \tab $pte.base_address$ = (fisica $>>$ 12);
    \State \tab \tab \tab }
  \State \tab \tab } else {
    \State \tab \tab \tab unsigned int $proxima_pag$ = $mmu_proxima_pagina_fisica_libre$();
    \State \tab \tab \tab pde.present = 1;
    \State \tab \tab \tab $pde.read_write$ = $read_write$;
    \State \tab \tab \tab $pde.user_supervisor$ = $user_supervisor$;
    \State \tab \tab \tab $pde.base_address$ = $proxima_pag >>$ 12;
    \State \tab \tab \tab $page_table_entry$* pt = ($page_table_entry$*)($proxima_pag$);
    \State \tab \tab \tab pt[indiceTable].present = 1;
    \State \tab \tab \tab pt[indiceTable].$user_supervisor$ = $user_supervisor$;
    \State \tab \tab \tab pt[indiceTable].$read_write$ = $read_write$;
    \State \tab \tab \tab pt[indiceTable].$base_address$ = (fisica $>>$ 12);

\State \tab \tab  }

  \State \tab \tab tlbflush();
\State \tab }

    
    
\end{algorithmic}

Finalmente, mara unmapear una pagina, lo hacemos de la siguiente manera:
\begin{algorithmic}
     
\State \tab void $mmu_unmapear_pagina$(unsigned int virtual, unsigned int cr3) {
  \State \tab \tab $page_directory_entry$* pd = ($page_directory_entry$*)(cr3);
  \State \tab \tab unsigned int indiceDirectory = virtual $>>$ 22;
  \State \tab \tab unsigned int indiceTable = (virtual $>>$ 12) $<<$ 10;
  \State \tab \tab $page_directory_entry$ pde = pd[indiceDirectory];
  \State \tab \tab $page_table_entry$* pt = ($page_table_entry$*)(pde.$base_address <<$ 12);
  \State \tab \tab pt[indiceTable].present = 0;
  \State \tab \tab int i = 0;
  \State \tab \tab int estanTodasEnNotPresent = 1;
  \State \tab \tab while (i $<$ 1024 $\&\&$ estanTodasEnNotPresent == 1) {
    \State \tab \tab \tab $page_table_entry$* pt = ($page_table_entry$*)(pde.$base_address <<$ 12);
    \State \tab \tab \tab $page_table_entry$ pte = pt[i];
    \State \tab \tab \tab if (pte.present == 1) {
      \State \tab \tab \tab \tab estanTodasEnNotPresent = 0;
    \State \tab \tab \tab }
    \State \tab \tab \tab i++;
  \State \tab \tab }
  \State \tab \tab if (estanTodasEnNotPresent == 1) {
    \State \tab \tab \tab pde.present = 0;
  \State \tab \tab  }

  \State \tab \tab tlbflush();
\State \tab  }
\end{algorithmic}

Durante el juego, debemos poder paginar de manera din\'amica las direcciones de los piratas que se agregan, para poder realizarlo utilizamos el siguiente m\'etodo:
\begin{algorithmic}
    \State \tab unsigned int $mmu_proxima_pagina_fisica_libre$() {
  \State \tab \tab unsigned int $pagina_libre$ = $proxima_pagina_libre$;
  \State \tab \tab $proxima_pagina_libre$ += $PAGE_SIZE$;
  \State \tab \tab return $pagina_libre$;
\State \tab }
\end{algorithmic}
















\begin{algorithmic}
\State \tab unsigned $ int mmu_inicializar_dir_pirata(jugador_t* jugador, pirata_t* tarea) \{$
  //aca ya tengo la page directory y table de la tarea.

\State \tab \tab  $ page_directory_entry* pd = (page_directory_entry*) mmu_proxima_pagina_fisica_libre();$
\State \tab \tab  $ page_table_entry* pt = (page_table_entry*) mmu_proxima_pagina_fisica_libre();$

\State \tab \tab  $ pd[0].present = 1;$
\State \tab \tab  $ pd[0].read_write = 1;$
\State \tab \tab  $ pd[0].user_supervisor = 0;$
\State \tab \tab  $ pd[0].write_through = 0;$
\State \tab \tab  $ pd[0].cache_disable = 0;$
\State \tab \tab  $ pd[0].accessed = 0;$
\State \tab \tab  $ pd[0].ignored = 0;$
\State \tab \tab  $ pd[0].page_size = 0;$
\State \tab \tab  $ pd[0].global = 0;$
\State \tab \tab  $ unsigned int page_table_address = (unsigned int)pt;$
\State \tab \tab  $ pd[0].base_address = page_table_address >> 12;$
\State \tab \tab  $ for (int i = 0; i < 1024; i++) \{$
 \State \tab \tab \tab $    pt[i].present = 1;$
 \State \tab \tab \tab $    pt[i].read_write = 1;$
 \State \tab \tab \tab $    pt[i].user_supervisor = 0;$
 \State \tab \tab \tab $    pt[i].write_through = 0;$
 \State \tab \tab \tab $    pt[i].cache_disable = 0;$
 \State \tab \tab \tab $    pt[i].accessed = 0;$
 \State \tab \tab \tab $    pt[i].dirty = 0;$
 \State \tab \tab \tab $    pt[i].attribute_index = 0;$
 \State \tab \tab \tab $    pt[i].global = 0;$
 \State \tab \tab \tab $   pt[i].base_address = i;$
\State \tab \tab $  \}$
\State \tab \tab  $ unsigned int page_directory_address = (unsigned int)pd;$
\State \tab \tab  $ if (jugador->index == JUGADOR_A) \{$
    //empiezo en la primera posicion
    //mapeo la 0x400000
  \State \tab \tab \tab $   mmu_mapear_pagina((unsigned int)0x00400000, (page_directory_address), pos2mapFis(1,2), 1, 1);$
  \State \tab \tab \tab $   breakpoint();$
  \State \tab \tab \tab $   if (tarea->tipo == explorador) \{$
      \State \tab \tab \tab \tab $ for (int i = 0; i < 1024; i++) \{$
        \State \tab \tab \tab \tab \tab $ ((unsigned int*)((unsigned int)0x00400000))[i] = ((unsigned int*)((unsigned int)0x10000))[i];$
      \State \tab \tab \tab \tab $ \}$
   \State \tab \tab \tab $  \} else \{$
      \State \tab \tab \tab \tab $ for (int i = 0; i < 1024; i++) \{$
        \State \tab \tab \tab \tab \tab $ ((unsigned int*)((unsigned int)0x00400000))[i] = ((unsigned int*)((unsigned int)0x11000))[i];$
      \State \tab \tab \tab \tab $ \}$
       \State \tab \tab \tab $ \} $
  \State \tab \tab \tab $   breakpoint();$
    //mapeo donde estamos parados
  \State \tab \tab \tab $   mmu_mapear_pagina(pos2mapVir(1,2), (page_directory_address), pos2mapFis(1,2), 1, 1);$
  \State \tab \tab \tab $   if (tarea->tipo == explorador) \{$
      //explorador
      \State \tab \tab \tab \tab $ for (int i = 0; i < 1024; i++) \{$
        ((unsigned int*)(pos2mapVir(1,2)))[i] = ((unsigned int*)((unsigned int)0x10000))[i];
      \State \tab \tab \tab \tab $ \}$
  \State \tab \tab \tab $   \} else \{$
      //minero
      \State \tab \tab \tab \tab $ for (int i = 0; i < 1024; i++) \{$
        ((unsigned int*)(pos2mapVir(1,2)))[i] = ((unsigned int*)((unsigned int)0x11000))[i];
     \State \tab \tab \tab \tab $  \}$
  \State \tab \tab \tab $   \}$

  \State \tab \tab \tab $   mmu_unmapear_pagina(pos2mapVir(1,2), (page_directory_address));$
\State \tab \tab \tab $ $
    //se mapean las de alrededor para jugador 1
  \State \tab \tab \tab $   mmu_mapear_pagina(pos2mapVir(2,1), (page_directory_address), pos2mapFis(2,1), 0, 1);//derecha$
  \State \tab \tab \tab $   mmu_mapear_pagina(pos2mapVir(2,2), (page_directory_address), pos2mapFis(2,2), 0, 1);//abajo-derecha$
  \State \tab \tab \tab $   mmu_mapear_pagina(pos2mapVir(1,3), (page_directory_address), pos2mapFis(1,3), 0, 1);//abajo$
  \State \tab \tab \tab $   mmu_mapear_pagina(pos2mapVir(0,1), (page_directory_address), pos2mapFis(0,1), 0, 1);//izquierda$
  \State \tab \tab \tab $   mmu_mapear_pagina(pos2mapVir(0,2), (page_cdirectory_address), pos2mapFis(0,2), 0, 1);//abajo-izquierda$
  \State \tab \tab \tab $   mmu_mapear_pagina(pos2mapVir(1,1), (page_directory_address), pos2mapFis(1,1), 0, 1);//arriba$
  \State \tab \tab \tab $   mmu_mapear_pagina(pos2mapVir(0,3), (page_directory_address), pos2mapFis(0,3), 0, 1);//arriba-izquierda$
  \State \tab \tab \tab $   mmu_mapear_pagina(pos2mapVir(2,3), (page_directory_address), pos2mapFis(2,3), 0, 1);//arriba-derecha$
 \State \tab \tab $ \} else \{$
    //empiezo en la ultima posicion
  \State \tab \tab \tab $   breakpoint();$
    //mapeo la 0x400000
  \State \tab \tab \tab $   mmu_mapear_pagina((unsigned int)0x00400000, (page_directory_address), pos2mapFis(1,2), 1, 1);$
  \State \tab \tab \tab $   if (tarea->tipo == explorador) \{$
      \State \tab \tab \tab \tab $ for (int i = 0; i < 1024; i++) \{$
        ((unsigned int*)((unsigned int)0x00400000))[i] = ((unsigned int*)((unsigned int)0x12000))[i];
      \State \tab \tab \tab \tab $ \}$
  \State \tab \tab \tab $   \} else \{$
      \State \tab \tab \tab \tab $  for (int i = 0; i < 1024; i++) \{$
        \State \tab \tab \tab \tab $  \tab ((unsigned int*)((unsigned int)0x00400000))[i] = ((unsigned int*)((unsigned int)0x13000))[i];$
      \State \tab \tab \tab \tab $ \}$
    \State \tab \tab \tab $ \}$
  \State \tab \tab \tab $   breakpoint();$
    //mapeo donde estamos parados
  \State \tab \tab \tab $   mmu_mapear_pagina(pos2mapVir(78,43), (page_directory_address), pos2mapFis(78,43), 0, 1);$
  \State \tab \tab \tab $   if (tarea->tipo == explorador) \{$
      //explorador
     \State \tab \tab \tab \tab $  for (int i = 0; i < 1024; i++) \{$
        ((unsigned int*)(pos2mapVir(78,43)))[i] = ((unsigned int*)((unsigned int)0x12000))[i];
      \State \tab \tab \tab \tab $ \}$
  \State \tab \tab \tab $  \} else \{$
      //minero
     \State \tab \tab \tab \tab $  for (int i = 0; i < 1024; i++) \{$
       \State \tab \tab \tab \tab $  \tab ((unsigned int*)(pos2mapVir(78,43)))[i] = ((unsigned int*)((unsigned int)0x13000))[i];$
    \State \tab \tab \tab \tab $   \}$
  \State \tab \tab \tab $   \}$

  \State \tab \tab \tab $   mmu_unmapear_pagina(pos2mapVir(78,43), (page_directory_address));$

    //al principio solo se mapean las paginas de la izquierda, arriba, arriba-izquierda para jug 2
  \State \tab \tab \tab $   mmu_mapear_pagina(pos2mapVir(77,43), (page_directory_address), pos2mapFis(77,43), 0, 1);//izquierda$
  \State \tab \tab \tab $   mmu_mapear_pagina(pos2mapVir(77,42), (page_directory_address), pos2mapFis(77,42), 0, 1);//arriba-izquierda$
  \State \tab \tab \tab $   mmu_mapear_pagina(pos2mapVir(78,42), (page_directory_address), pos2mapFis(78,42), 0, 1);//arriba$
  \State \tab \tab \tab $   mmu_mapear_pagina(pos2mapVir(79,43), (page_directory_address), pos2mapFis(79,43), 0, 1);//derecha$
  \State \tab \tab \tab $   mmu_mapear_pagina(pos2mapVir(79,42), (page_directory_address), pos2mapFis(79,42), 0, 1);//arriba-derecha$
  \State \tab \tab \tab $   mmu_mapear_pagina(pos2mapVir(78,44), (page_directory_address), pos2mapFis(78,44), 0, 1);//abajo$
  \State \tab \tab \tab $   mmu_mapear_pagina(pos2mapVir(79,44), (page_directory_address), pos2mapFis(79,44), 0, 1);//abajo-derecha$
  \State \tab \tab \tab $   mmu_mapear_pagina(pos2mapVir(77,44), (page_directory_address), pos2mapFis(77,44), 0, 1);//abajo-izquierda$
\State \tab \tab $  \}$
\State \tab \tab  $ breakpoint();$
\State \tab \tab  $ return page_directory_address;$
\State \tab $ \}$
\end{algorithmic}


\subsection{Manejo de Interrupciones}
Para poder atender los distintos tipos de interrupciones, definimos las tareas de atenci\'on en \textit{isr.asm}
Definimos una rutina para atender las interrupciones del reloj:
\begin{algorithmic}
    \State \tab global $_isr32$
\State \tab $_isr32$:
  \State \tab \tab; PRESERVAR REGISTROS
  \State \tab \tab pushad
  \State \tab \tab call $fin_intr_pic1$
  \State \tab \tab cmp byte [modoDebug], 1
  \State \tab \tab je .fin
\State \tab \tab  call $sched_tick$
  \State \tab \tab str cx
  \State \tab \tab cmp ax, cx
  \State \tab \tab je .fin
  \State \tab \tab mov [$sched_tarea_selector$], ax
  \State \tab \tab jmp far [$sched_tarea_offset$]
\State \tab \tab .fin:
  \State \tab \tab ; RESTAURAR REGISTROS
  \State \tab popad
  \State \tab iret
\end{algorithmic}

Luego, para ateneder las interrupciones correspondientes al teclado lo hacemos de la siguiente manera:
\begin{algorithmic}
    \State \tab $global _isr33$
\State \tab \tab $_isr33$:
  \State \tab \tab pushad
  \State \tab \tab call $fin_intr_pic1$
  \State \tab \tab xor ax, ax
  \State \tab \tab in al, 0x60
  \State \tab \tab push eax
  \State \tab \tab call $game_atender_teclado$
  \State \tab \tab pop eax
  \State \tab \tab popad
  \State \tab iret
\end{algorithmic}

El m\'etodo \textit{$game_atender_teclado$} esta definido en nuestro archivo \textit{game.c}. Este m\'etodo lo que hace, es imprimir en el rincon derecho superior de la pantalla la tecla que se presion\'o.
Cuenta con un switch, que evalua el caso de cada tecla posible, y en base a esto imprime lo que corresponde.


Para atender las interrupciones por excepciones, definimos una macro:
\begin{algorithmic}
\State \tab $\%$ macro ISR 1
\State \tab global $\_$isr$\%1$

\State \tab $\_$ isr$\%$1:
    \State \tab \tab mov eax, $\%1$
    \State \tab \tab imprimir$\_$texto$\_$mp desc$\_\%$1, desc$\_$len$\_\%$1, 0x07, 3, 0
    \State \tab \tab jmp $\$$

\State \tab %endmacro
\end{algorithmic}

De esta manera, podemos definir el mensaje correspondiente para cada interrupcion, y utilizando la macro, no tenemos que repetir el c\'odigo, ya que para todas las interrupciones va a trabajar de la misma manera. Va a imprimir en pantalla el nombre de la interrupci\'on que acaba de ocurrir.
Por ejemplo, cuando queremos identificar la interrupci\'on de "Divide Error", definimos el mensaje de la siguiente maner:

\begin{algorithmic}
\State \tab desc$\_$0 db   'Divide Error'
\State \tab desc$\_$len$\_$0 equ     $\$$ - desc$\_$0
\end{algorithmic}

De esta misma manera definimos todos los mensajes correspondientes a las interrupciones, en nuestro archivo \textit{isr.asm}.
\subsection{Tareas}

Completamos las TSS utilizando c\'odigo c, en el archivo \textit{tss.c}
Para completar la \textit{idle}  y la \textit{inicial}:

\begin{algorithmic}
\State \tab void tss$\_$inicializar$() \{$\\
  \State \tab \tab //tss$\_$idle\\

  \State \tab \tab tss$\_$idle.ptl = 0;
  \State \tab \tab tss$\_$idle.unused0 = 0;
  \State \tab \tab tss$\_$idle.esp0 = 0;
  \State \tab \tab tss$\_$idle.ss0 = 0;
  \State \tab \tab tss$\_$idle.unused1 = 0;
  \State \tab \tab tss$\_$idle.esp1 = 0;
  \State \tab \tab tss$\_$idle.ss1 = 0;
  \State \tab \tab tss$\_$idle.unused2 = 0;
  \State \tab \tab tss$\_$idle.esp2 = 0;
  \State \tab \tab tss$\_$idle.ss2 = 0;
  \State \tab \tab tss$\_$idle.unused3 = 0;
  \State \tab \tab tss$\_$idle.cr3 = $($unsigned int$)$0x27000;
  \State \tab \tab tss$\_$idle.eip = $($unsigned int$)$0x16000;
  \State \tab \tab tss$\_$idle.eflags = $($unsigned int$)$0x00202;
  \State \tab \tab tss$\_$idle.eax = 0;
  \State \tab \tab tss$\_$idle.ecx = 0;
  \State \tab \tab tss$\_$idle.edx = 0;
  \State \tab \tab tss$\_$idle.ebx = 0;
  \State \tab \tab tss$\_$idle.esp = $($unsigned int$)$0x27000;
  \State \tab \tab tss$\_$idle.ebp = $($unsigned int$)$0x27000;
  \State \tab \tab tss$\_$idle.esi = 0;
  \State \tab \tab tss$\_$idle.edi = 0;
  \State \tab \tab tss$\_$idle.es = $($unsigned int$)$0x48;
  \State \tab \tab tss$\_$idle.unused4 = 0;
  \State \tab \tab tss$\_$idle.cs = $($unsigned int$)$0x58;
  \State \tab \tab tss$\_$idle.unused5 = 0;
  \State \tab \tab tss$\_$idle.ss = $($unsigned int$)$0x48;
  \State \tab \tab tss$\_$idle.unused6 = 0;
  \State \tab \tab tss$\_$idle.ds = $($unsigned int$)$0x48;
  \State \tab \tab tss$\_$idle.unused7 = 0;
  \State \tab \tab tss$\_$idle.fs = $($unsigned int$)$0x00060;
  \State \tab \tab tss$\_$idle.unused8 = 0;
  \State \tab \tab tss$\_$idle.gs = $($unsigned int$)$0x48;
  \State \tab \tab tss$\_$idle.unused9 = 0;
  \State \tab \tab tss$\_$idle.ldt = 0;
  \State \tab \tab tss$\_$idle.unused10 = 0;
  \State \tab \tab tss$\_$idle.dtrap = 0;
  \State \tab \tab tss$\_$idle.iomap = 0;\\

  \State \tab \tab //tss$\_$inicial\\

  \State \tab \tab tss$\_$inicial.ptl = 0;
  \State \tab \tab tss$\_$inicial.unused0 = 0;
  \State \tab \tab tss$\_$inicial.esp0 = 0;
  \State \tab \tab tss$\_$inicial.ss0 = 0;
  \State \tab \tab tss$\_$inicial.unused1 = 0;
  \State \tab \tab tss$\_$inicial.esp1 = 0;
  \State \tab \tab tss$\_$inicial.ss1 = 0;
  \State \tab \tab tss$\_$inicial.unused2 = 0;
  \State \tab \tab tss$\_$inicial.esp2 = 0;
  \State \tab \tab tss$\_$inicial.ss2 = 0;
  \State \tab \tab tss$\_$inicial.unused3 = 0;
  \State \tab \tab tss$\_$inicial.cr3 = 0;
  \State \tab \tab tss$\_$inicial.eip = 0;
  \State \tab \tab tss$\_$inicial.eflags = 0;
  \State \tab \tab tss$\_$inicial.eax = 0;
  \State \tab \tab tss$\_$inicial.ecx = 0;
  \State \tab \tab tss$\_$inicial.edx = 0;
  \State \tab \tab tss$\_$inicial.ebx = 0;
  \State \tab \tab tss$\_$inicial.esp = 0;
  \State \tab \tab tss$\_$inicial.ebp = 0;
  \State \tab \tab tss$\_$inicial.esi = 0;
  \State \tab \tab tss$\_$inicial.edi = 0;
  \State \tab \tab tss$\_$inicial.es = 0;
  \State \tab \tab tss$\_$inicial.unused4 = 0;
  \State \tab \tab tss$\_$inicial.cs = 0;
  \State \tab \tab tss$\_$inicial.unused5 = 0;
  \State \tab \tab tss$\_$inicial.ss = 0;
  \State \tab \tab tss$\_$inicial.unused6 = 0;
  \State \tab \tab tss$\_$inicial.ds = 0;
  \State \tab \tab tss$\_$inicial.unused7 = 0;
  \State \tab \tab tss$\_$inicial.fs = 0;
  \State \tab \tab tss$\_$inicial.unused8 = 0;
  \State \tab \tab tss$\_$inicial.gs = 0;
  \State \tab \tab tss$\_$inicial.unused9 = 0;
  \State \tab \tab tss$\_$inicial.ldt = 0;
  \State \tab \tab tss$\_$inicial.unused10 = 0;
  \State \tab \tab tss$\_$inicial.dtrap = 0;
  \State \tab \tab tss$\_$inicial.iomap = 0;
\State \tab $\}$
\end{algorithmic}


Debemos agregar las \textit{TSS} correspondientes a la tarea \textit{idle} y la tarea \textit{inicial} a nuestra \textit{GDT}, para poder realizar esto, definimos un m\'etodo llamado \textit{tss$\_$agregar$\_$a$\_$gdt}
El m\'etodo realiza lo siguiente:

\begin{algorithmic}
\State \tab  void tss$\_$agregar$\_$a$\_$gdt$() \{$
  \State \tab \tab gdt[GDT$\_$TSS$\_$IDLE] = $($gdt$\_$entry$)$ {
      \State \tab \tab \tab $($unsigned short$)$    0x0067,         /* limit[0:15]  */
      \State \tab \tab \tab $($unsigned short$)$    $($int$)$$($$\&$tss$\_$idle$)$ $\&$ 0xFFFF,         /* base[0:15]   */
      \State \tab \tab \tab $($unsigned char$)$     $($int$)$$($$($int$)$$($$\&$tss$\_$idle$)$ $\gg$ 16$)$ $\&$ 0x00FF,   /* base[23:16]  */
      \State \tab \tab \tab $($unsigned char$)$     0x09,           /* type         */
      \State \tab \tab \tab $($unsigned char$)$     0x00,           /* s            */
      \State \tab \tab \tab $($unsigned char$)$     0x00,           /* dpl          */
      \State \tab \tab \tab $($unsigned char$)$     0x01,           /* p            */
      \State \tab \tab \tab $($unsigned char$)$     0x00,           /* limit[16:19] */
      \State \tab \tab \tab $($unsigned char$)$     0x00,           /* avl          */
      \State \tab \tab \tab $($unsigned char$)$     0x00,           /* l            */
      \State \tab \tab \tab $($unsigned char$)$     0x00,           /* db           */
      \State \tab \tab \tab $($unsigned char$)$     0x00,           /* g            */
      \State \tab \tab \tab $($unsigned char$)$     $($int$)$$($$\&$tss$\_$idle$)$ $\gg$ 24,           /* base[31:24]  */
  \State \tab \tab };
  \State \tab \tab gdt[GDT$\_$TSS$\_$INICIAL] = $($gdt$\_$entry$)$ {
      \State \tab \tab \tab $($unsigned short$)$    0x0067,         /* limit[0:15]  */
      \State \tab \tab \tab $($unsigned short$)$    $($int$)$$($$\&$tss$\_$inicial$)$ $\&$ 0xFFFF, /* base[0:15]   */
      \State \tab \tab \tab $($unsigned char$)$     $($int$)$$($$($int$)$$($$\&$tss$\_$inicial$)$ $\gg$ 16$)$ $\&$ 0x00FF,           /* base[23:16]  */
      \State \tab \tab \tab $($unsigned char$)$     0x09,           /* type         */
      \State \tab \tab \tab $($unsigned char$)$     0x00,           /* s            */
      \State \tab \tab \tab $($unsigned char$)$     0x00,           /* dpl          */
      \State \tab \tab \tab $($unsigned char$)$     0x01,           /* p            */
      \State \tab \tab \tab $($unsigned char$)$     0x00,           /* limit[16:19] */
      \State \tab \tab \tab $($unsigned char$)$     0x00,           /* avl          */
      \State \tab \tab \tab $($unsigned char$)$     0x00,           /* l            */
      \State \tab \tab \tab $($unsigned char$)$     0x00,           /* db           */
      \State \tab \tab \tab $($unsigned char$)$     0x00,           /* g            */
      \State \tab \tab \tab $($unsigned char$)$     $($int$)$$($$\&$tss$\_$inicial$)$ $\gg$ 24,           /* base[31:24]  */
  \State \tab \tab };
\State \tab $\}$
\end{algorithmic}

Luego necesitamos completar la TSS correspondiente a cada pirata, para esto realizamos el siguiente m\'etodo:

\begin{algorithmic}
\State \tab void completarTssPirata$($pirata$\_$t tarea$) \{$
  \State \tab \tab unsigned int paginaParaPilaCero = mmu$\_$proxima$\_$pagina$\_$fisica$\_$libre$()$ + 0x1000;
\State \tab \tab 
  \State \tab \tab tss* tss$\_$pirata = $($*$($tarea.jugador$)$$)$.index == JUGADOR$\_$A ? $\&$tss$\_$jugadorA[tarea.id] : $\&$tss$\_$jugadorB[tarea.id];
\State \tab \tab 
  \State \tab \tab tss$\_$pirata$\to$eip = 0x00400000;
  \State \tab \tab tss$\_$pirata$\to$ptl = 0;
  \State \tab \tab tss$\_$pirata$\to$unused0 = 0;
  \State \tab \tab tss$\_$pirata$\to$esp0 = paginaParaPilaCero;
  \State \tab \tab tss$\_$pirata$\to$ss0 = $($unsigned int$)$0x40;
  \State \tab \tab tss$\_$pirata$\to$unused1 = 0;
  \State \tab \tab tss$\_$pirata$\to$esp1 = 0;
  \State \tab \tab tss$\_$pirata$\to$ss1 = 0;
  \State \tab \tab tss$\_$pirata$\to$unused2 = 0;
  \State \tab \tab tss$\_$pirata$\to$esp2 = 0;
  \State \tab \tab tss$\_$pirata$\to$ss2 = 0;
  \State \tab \tab tss$\_$pirata$\to$unused3 = 0;
  \State \tab \tab tss$\_$pirata$\to$cr3 = mmu$\_$inicializar$\_$dir$\_$pirata$($tarea.jugador, $\&$tarea$)$;
  \State \tab \tab tss$\_$pirata$\to$eflags = $($unsigned int$)$0x00202;
  \State \tab \tab tss$\_$pirata$\to$eax = 0;
  \State \tab \tab tss$\_$pirata$\to$ecx = 0;
  \State \tab \tab tss$\_$pirata$\to$edx = 0;
  \State \tab \tab tss$\_$pirata$\to$ebx = 0;
  \State \tab \tab tss$\_$pirata$\to$esp = 0x00401000 - 12;
  \State \tab \tab tss$\_$pirata$\to$ebp = 0x00401000 - 12;
  \State \tab \tab tss$\_$pirata$\to$esi = 0;
  \State \tab \tab tss$\_$pirata$\to$edi = 0;
  \State \tab \tab tss$\_$pirata$\to$es = $($unsigned int$)$0x40 | 0x3;
  \State \tab \tab tss$\_$pirata$\to$unused4 = 0;
  \State \tab \tab tss$\_$pirata$\to$cs = $($unsigned int$)$0x50 | 0x3;
  \State \tab \tab tss$\_$pirata$\to$unused5 = 0;
  \State \tab \tab tss$\_$pirata$\to$ss = $($unsigned int$)$0x40 | 0x3;
  \State \tab \tab tss$\_$pirata$\to$unused6 = 0;
  \State \tab \tab tss$\_$pirata$\to$ds = $($unsigned int$)$0x40 | 0x3;
  \State \tab \tab tss$\_$pirata$\to$unused7 = 0;
  \State \tab \tab tss$\_$pirata$\to$fs = $($unsigned int$)$0x40 | 0x3;
  \State \tab \tab tss$\_$pirata$\to$unused8 = 0;
  \State \tab \tab tss$\_$pirata$\to$gs = $($unsigned int$)$0x40 | 0x3;
  \State \tab \tab tss$\_$pirata$\to$unused9 = 0;
  \State \tab \tab tss$\_$pirata$\to$ldt = 0;
  \State \tab \tab tss$\_$pirata$\to$unused10 = 0;
  \State \tab \tab tss$\_$pirata$\to$dtrap = 0;
  \State \tab \tab tss$\_$pirata$\to$iomap = 0xFFFF;
\State \tab $\}$
\end{algorithmic}

Al igual que hicimos con la tarea \textit{idle} y la tarea \textit{inicial}, debemos agregar las tareas correspondientes a los piratas a nuestra \textit{GDT}, para esto desarrollamos el m\'etodo \textit{tss$\_$agretar$\_$piratas$\_$a$\_$gdt}

\begin{algorithmic}
\State \tab  void tss$\_$agregar$\_$piratas$\_$a$\_$gdt$($jugador$\_$t* j$)$ $\{$
\State \tab \tab    if $($j$\to$index == 0$)$  $\{$
\State \tab \tab      gdt[EMPIEZAN$\_$TSS + proximaTareaA] = $($gdt$\_$entry$)$ $\{$
\State \tab \tab \tab       $($unsigned short$)$    0x0067,         /* limit[0:15]  */
\State \tab \tab \tab       $($unsigned short$)$    $($int$)$$($$\&$tss$\_$jugadorA[jugadorA.piratas[proximaTareaA].index]$)$ $\&$ 0xFFFF, /* base[0:15]   */
\State \tab \tab \tab       $($unsigned char$)$     $($int$)$$($$($int$)$$($$\&$tss$\_$jugadorA[jugadorA.piratas[proximaTareaA].index]$)$ $\gg$ 16$)$ $\&$ 0x00FF,           /* base[23:16]  */
\State \tab \tab \tab       $($unsigned char$)$     0x09,           /* type         */
\State \tab \tab \tab       $($unsigned char$)$     0x00,           /* s            */
\State \tab \tab \tab       $($unsigned char$)$     0x03,           /* dpl          */
\State \tab \tab \tab       $($unsigned char$)$     0x01,           /* p            */
\State \tab \tab \tab       $($unsigned char$)$     0x00,           /* limit[16:19] */
\State \tab \tab \tab       $($unsigned char$)$     0x00,           /* avl          */
\State \tab \tab \tab       $($unsigned char$)$     0x00,           /* l            */
\State \tab \tab \tab       $($unsigned char$)$     0x00,           /* db           */
\State \tab \tab \tab       $($unsigned char$)$     0x00,           /* g            */
\State \tab \tab \tab       $($unsigned char$)$     $($int$)$$($$\&$tss$\_$jugadorA[jugadorA.piratas[proximaTareaA].index]$)$ $\gg$ 24,           /* base[31:24]  */
\State \tab \tab \tab     $\}$;
\State \tab \tab \tab     completarTssPirata$($jugadorA.piratas[proximaTareaA]$)$;
\State \tab \tab \tab   $\}$ else $\{$
\State \tab \tab \tab     gdt[EMPIEZAN$\_$TSS + 8 + proximaTareaB] = $($gdt$\_$entry$)$ $\{$
\State \tab \tab \tab       $($unsigned short$)$    0x0067,         /* limit[0:15]  */
\State \tab \tab \tab       $($unsigned short$)$    $($int$)$$($$\&$tss$\_$jugadorB[jugadorB.piratas[proximaTareaB].index]$)$ $\&$ 0xFFFF, /* base[0:15]   */
\State \tab \tab \tab       $($unsigned char$)$     $($int$)$$($$($int$)$$($$\&$tss$\_$jugadorB[jugadorB.piratas[proximaTareaB].index]$)$ $\gg$ 16$)$ $\&$ 0x00FF,           /* base[23:16]  */
\State \tab \tab \tab       $($unsigned char$)$     0x09,           /* type         */
\State \tab \tab \tab       $($unsigned char$)$     0x00,           /* s            */
\State \tab \tab \tab       $($unsigned char$)$     0x03,           /* dpl          */
\State \tab \tab \tab       $($unsigned char$)$     0x01,           /* p            */
\State \tab \tab \tab       $($unsigned char$)$     0x00,           /* limit[16:19] */
\State \tab \tab \tab       $($unsigned char$)$     0x00,           /* avl          */
\State \tab \tab \tab       $($unsigned char$)$     0x00,           /* l            */
\State \tab \tab \tab       $($unsigned char$)$     0x00,           /* db           */
\State \tab \tab \tab       $($unsigned char$)$     0x00,           /* g            */
\State \tab \tab \tab       $($unsigned char$)$     $($int$)$$($$\&$tss$\_$jugadorB[jugadorB.piratas[proximaTareaB].id]$)$ $\gg$ 24,           /* base[31:24]  */
\State \tab \tab \tab     $\}$;
\State \tab \tab \tab     completarTssPirata$($jugadorB.piratas[proximaTareaB]$)$;
\State \tab \tab $\}$
\State \tab $\}$
\end{algorithmic}

\subsection{Scheduler}

Decidimos representar al \textit{Scheduler} de la siguiente manera:
\begin{algorithmic}
\State \tab uint proximaTareaA; //indice 0-7
\State \tab uint proximaTareaB; //indice 0-7
\State \tab uchar turnoPirata; //0 A, 1 B
\State \tab uchar estaEnIdle; // 0 NO, 1 SI
\State \tab uint modoDebug;
     
\end{algorithmic}

Para inicializar estos valores, definimos el siguiente m\'etodo:
\begin{algorithmic}
    \State \tab void sched$\_$inicializar$() \{$
    \State \tab \tab    turnoPirata = 0;
    \State \tab \tab    proximaTareaA = 0;
    \State \tab \tab estaEnIdle = 1;
    \State \tab \tab proximaTareaB = 0;
    \State \tab \tab modoDebug = 0;
    \State \tab $\}$
\end{algorithmic}


Para atender a la proxima tarea, vamos a reutilizar la funci\'on que cambia las tareas de acuerdo al tick del clock, utilizamos a la funci\'on:
\begin{algorithmic}
    \State \tab unsigned int sched$\_$proxima$\_$a$\_$ejecutar$() \{$
    \State \tab \tab return sched$\_$tick$()$;
    \State \tab $\}$
\end{algorithmic}

Luego, para realizar lo mencionado anteriormente, correspondiente al tick del clock:

\begin{algorithmic}
\State \tab unsigned int sched$\_$tick$()$ $\{$
\State \tab \tab  if $($estaEnIdle == 1$)$ $\{$
\State \tab \tab  \tab    estaEnIdle = 0;
\State \tab \tab  \tab    if $($turnoPirata == 0$)$ $\{$
\State \tab \tab  \tab \tab      //turno jug A
\State \tab \tab  \tab \tab      uint proxTarea = EMPIEZAN$\_$TSS + proximaTareaA;
\State \tab \tab  \tab \tab      game$\_$tick$($proxTarea$)$;
\State \tab \tab  \tab \tab      turnoPirata = 1;
\State \tab \tab  \tab \tab      uchar noEncontreNinguna = 1;
\State \tab \tab  \tab \tab      uchar todosMuertos = 0;
\State \tab \tab  \tab \tab      int i = proximaTareaA + 1;
\State \tab \tab  \tab \tab      while $($noEncontreNinguna == 1 $\&\&$ todosMuertos == 0$)$ $\{$
    \State \tab \tab  \tab \tab \tab    if $($jugadorA.piratas[i].vivoMuerto == 1$)$ $\{$
          \State \tab \tab  \tab \tab \tab \tab //si esta vivo la pongo como la proxima tarea de A
          \State \tab \tab  \tab \tab \tab \tab proximaTareaA = jugadorA.piratas[i].index;
          \State \tab \tab  \tab \tab \tab \tab noEncontreNinguna = 0;
    \State \tab \tab  \tab \tab \tab    $\}$

    \State \tab \tab  \tab \tab \tab    if $($i == proximaTareaA$)$ $\{$
          todosMuertos = 1;
    \State \tab \tab  \tab \tab \tab    $\}$

    \State \tab \tab  \tab \tab \tab    i++;

    \State \tab \tab  \tab \tab \tab    if $($i == 8$)$ $\{$
         \State \tab \tab  \tab \tab \tab \tab  i = 0;
    \State \tab \tab  \tab \tab \tab    $\}$
\State \tab \tab  \tab \tab      $\}$

\State \tab \tab  \tab \tab      if $($todosMuertos$)$ $\{$
        //salto a la idle
      \State \tab \tab  \tab \tab \tab  return $($13$)$ $\ll$ 3;
\State \tab \tab  \tab \tab      $\}$

\State \tab \tab  \tab \tab      return $($proxTarea $\ll$ 3$)$;
\State \tab \tab  \tab    $\}$ else $\{$
 \State \tab \tab  \tab \tab     //turno jug B
 \State \tab \tab  \tab \tab     uint proxTarea = EMPIEZAN$\_$TSS + 8 + proximaTareaB;
 \State \tab \tab  \tab \tab     game$\_$tick$($proxTarea$)$;
 \State \tab \tab  \tab \tab     turnoPirata = 0;
 \State \tab \tab  \tab \tab     uchar noEncontreNinguna = 1;
 \State \tab \tab  \tab \tab     uchar todosMuertos = 0;
 \State \tab \tab  \tab \tab     int i = proximaTareaB + 1;
 \State \tab \tab  \tab \tab     while $($noEncontreNinguna == 1 $\&\&$ todosMuertos == 0$)$ $\{$
   \State \tab \tab  \tab \tab \tab     if $($jugadorB.piratas[i].vivoMuerto == 1$)$ $\{$
          //si esta vivo la pongo como la proxima tarea de B
          proximaTareaB = jugadorB.piratas[i].index;
          noEncontreNinguna = 0;
   \State \tab \tab  \tab \tab \tab     $\}$

   \State \tab \tab  \tab \tab \tab     if $($i == proximaTareaA$)$ $\{$
          \State \tab \tab  \tab \tab \tab \tab todosMuertos = 1;
   \State \tab \tab  \tab \tab \tab     $\}$

   \State \tab \tab  \tab \tab \tab     i++;

   \State \tab \tab  \tab \tab \tab     if $($i == 8$)$ $\{$
         \State \tab \tab  \tab \tab \tab \tab  i = 0;
   \State \tab \tab  \tab \tab \tab     $\}$
  \State \tab \tab  \tab \tab    $\}$

  \State \tab \tab  \tab \tab    if $($todosMuertos$)$ $\{$
    \State \tab \tab  \tab \tab \tab    //salto a la idle
    \State \tab \tab  \tab \tab \tab    return $($13$)$ $\ll$ 3;
  \State \tab \tab  \tab \tab    $\}$

  \State \tab \tab  \tab \tab    return $($proxTarea $\ll$ 3$)$;
\State \tab \tab  \tab    $\}$
\State \tab \tab    $\}$ else $\{$
\State \tab \tab  \tab    if $($turnoPirata == 0$)$ $\{$
     \State \tab \tab  \tab \tab  //turno proximo es A
      \State \tab \tab  \tab \tab  uint proxTarea = EMPIEZAN$\_$TSS + proximaTareaA;
      \State \tab \tab  \tab \tab  game$\_$tick$($proxTarea$)$;
      \State \tab \tab  \tab \tab  turnoPirata = 1;
      \State \tab \tab  \tab \tab  uchar noEncontreNinguna = 1;
      \State \tab \tab  \tab \tab  uchar todosMuertos = 0;
      \State \tab \tab  \tab \tab  int i = proximaTareaA + 1;
      \State \tab \tab  \tab \tab  while $($noEncontreNinguna == 1 $\&\&$ todosMuertos == 0$)$ $\{$
   \State \tab \tab  \tab \tab \tab     if $($jugadorA.piratas[i].vivoMuerto == 1$)$ $\{$
         \State \tab \tab  \tab \tab  \tab \tab //si esta vivo la pongo como la proxima tarea de A
         \State \tab \tab  \tab \tab \tab \tab proximaTareaA = jugadorA.piratas[i].index;
          \State \tab \tab  \tab \tab \tab \tab  noEncontreNinguna = 0;
   \State \tab \tab  \tab \tab \tab     $\}$

    \State \tab \tab  \tab \tab \tab    if $($i == proximaTareaA$)$ $\{$
         \State \tab \tab  \tab \tab \tab \tab todosMuertos = 1;
    \State \tab \tab  \tab \tab \tab    $\}$

    \State \tab \tab  \tab \tab \tab    i++;

    \State \tab \tab  \tab \tab \tab    if $($i == 8$)$ $\{$
        \State \tab \tab  \tab \tab \tab \tab  i = 0;
    \State \tab \tab  \tab \tab \tab    $\}$
    \State \tab \tab  \tab \tab    $\}$

      \State \tab \tab  \tab \tab  if $($todosMuertos$)$ $\{$
        //salto a la idle
        \State \tab \tab  \tab \tab \tab return $($13$)$ $\ll$ 3;
      \State \tab \tab  \tab \tab  $\}$

      \State \tab \tab  \tab \tab  return $($proxTarea$)$ $\ll$ 3;
   \State \tab \tab  \tab $\}$ else $\{$
      \State \tab \tab  \tab \tab  //turno proximo es B
      \State \tab \tab  \tab \tab  uint proxTarea = EMPIEZAN$\_$TSS + 8 + proximaTareaB;
      \State \tab \tab  \tab \tab  game$\_$tick$($proxTarea$)$;
      \State \tab \tab  \tab \tab  turnoPirata = 0;
      \State \tab \tab  \tab \tab  uchar noEncontreNinguna = 1;
      \State \tab \tab  \tab \tab  uchar todosMuertos = 0;
      \State \tab \tab  \tab \tab  int i = proximaTareaB + 1;
      \State \tab \tab  \tab \tab  while $($noEncontreNinguna == 1 $\&\&$ todosMuertos == 0$)$ $\{$
    \State \tab \tab  \tab \tab \tab    if $($jugadorB.piratas[i].vivoMuerto == 1$)$ $\{$
          //si esta vivo la pongo como la proxima tarea de B
          \State \tab \tab  \tab \tab \tab \tab proximaTareaB = jugadorB.piratas[i].index;
          \State \tab \tab  \tab \tab \tab \tab noEncontreNinguna = 0;
    \State \tab \tab  \tab \tab \tab    $\}$

    \State \tab \tab  \tab \tab \tab    if $($i == proximaTareaA$)$ $\{$
          \State \tab \tab  \tab \tab \tab \tab todosMuertos = 1;
    \State \tab \tab  \tab \tab \tab    $\}$

    \State \tab \tab  \tab \tab \tab    i++;

     \State \tab \tab  \tab \tab \tab   if $($i == 8$)$ $\{$
          \State \tab \tab  \tab \tab \tab \tab i = 0;
     \State \tab \tab  \tab \tab \tab   $\}$
      \State \tab \tab  \tab \tab $\}$
      \State \tab \tab  \tab \tab if $($todosMuertos$)$ $\{$
        //salto a la idle
        \State \tab \tab  \tab \tab \tab return $($13$)$ $\ll$ 3;
      \State \tab \tab  \tab \tab $\}$
      \State \tab \tab  \tab \tab return $($proxTarea$)$ $\ll$ 3;
\State \tab \tab  \tab    $\}$
\State \tab \tab    $\}$
\State \tab  $\}$
\end{algorithmic}

Por \'ultimo, otros m\'etodos que definimos en nuestro \textit{sched.c}
\begin{algorithmic}
    \State \tab void sched$\_$intercambiar$\_$por$\_$idle$() \{$
        \State \tab \tab estaEnIdle = 1;
    \State \tab $\}$     
\end{algorithmic}

\begin{algorithmic}
    \State \tab void sched$\_$nointercambiar$\_$por$\_$idle$() \{$
        \State \tab \tab estaEnIdle = 0;
    \State \tab $\}$     
\end{algorithmic}

\begin{algorithmic}
    \State \tab void sched$\_$toggle$\_$debug$() \{$
        \State \tab \tab if (modoDebug) $\{$
        \State \tab \tab \tab modoDebug = FALSE;
        \State \tab \tab $\}$ else $\{$
        \State \tab \tab \tab modoDebug = TRUE;
        \State \tab \tab$\}$
    \State \tab $\}$     
\end{algorithmic}
