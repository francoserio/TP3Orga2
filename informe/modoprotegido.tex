El \textit{Modo Protegido}, es un poco mas completo que el \textit{Modo Real}, tenemos 4GB de memoria disponible, 4 niveles de protecci\'on de memoria y rutinas de atenci\'on con privilegios.

\subsection{Segmentacion}
Para el mecanismo de segmentación necesitamos inicializar la GDT.
El primer descriptor de la tabla siempre es \textit{nulo}. También al principio tenemos otros 4 segmentos. 1 de código nivel kernel, 1 de código nivel usuario, 1 de datos nivel usuario y 1 de datos nivel kernel. Los segmentos de user tienen dpl 3 y los de sistema 0. A su vez, el límite de los segmentos de la gdt se saca porque en el TP dice que se direccionan los primeros 500MB de memoria. Eso significa que tenemos que usar granularidad de a 4KB de lo contrario no llegamos a los 500MB pedidos. El limite se calcula de hacer (500MB / 4KB) - 1. Esto es porque granularity como dijimos está seteado. El dpl es 3 si se trata de usuario y 0 si se trata de kernel. El tipo es 10 si se trata de  de código y 2 si se trata de datos. Hay otro descriptor al comienzo de nuestro sistema para el video. Aquí la base es 0xB8000 a diferencia del cero de las demás entradas. Ésto lo dice la página 2 del enunciado del TP. Todos los segmentos se ponen en presentes.
\newline
Nos encargamos de setear los offset de los descriptores de los selectores de segmentos. Luego seteamos la pila del kernel en la direcci\'on 0x27000 que indica el enunciado del TP. Al primer ax le cargamos el indice de la gdt de los datos de nivel 0 y al segundo el indice de la gdt del segmento de video, que son el 9 y el 12 respectivamente.
Todos los índices de la GDT se encuentran en \textbf{defines.h}. GDT$\_$count es 31 ya que hay 31 descriptores en la GDT y los índices más grandes que el 0 (descriptor NULL) están a partir del 8 por restricción en el enunciado de TP.

\begin{algorithmic}
\State \tab BITS 32
\State \tab mp:

    \State \tab \tab ; Establecer selectores de segmentos
    \State \tab \tab xor eax, eax
    \State \tab \tab  mov ax, 1001000b
    \State \tab \tab  mov ds, ax
    \State \tab \tab  mov ss, ax
    \State \tab \tab  mov es, ax
    \State \tab \tab  mov gs, ax
    \State \tab \tab  mov ax, 1100000b
    \State \tab \tab  mov fs, ax
    \State \tab \tab  ; Establecer la base de la pila
    \State \tab \tab  mov esp, 0x27000
    \State \tab \tab  mov ebp, 0x27000
\end{algorithmic}

Luego de esto imprimimos en pantalla un mensaje de bienvenida a nuestro sistema e inicializamos la pantalla y el juego.
\begin{algorithmic}
    \State \tab \tab ; Imprimir mensaje de bienvenida
    \State \tab \tab $imprimir\_texto\_mp \ iniciando\_mp\_msg, iniciando\_mp\_len$, 0x07, 2, 0
    \State \tab \tab call $game\_inicializar$
    \State \tab \tab call $screen\_inicializar$
    \State \tab \tab call $screen\_pintar\_nombre$
    \State \tab \tab call $screen\_pintar\_puntajes$
\end{algorithmic}

Las funciones \textit{$screen\_inicializar$}, \textit{$screen\_pintar\_nombre$} estan definidas en \textbf{screen.h} y desarrolladas en \textbf{screen.c}. Estos m\'etodos se encargan de pintar la pantalla, seg\'un lo pedido en el enunciado. La función \textit{$game\_inicializar$} esta definida en \textbf{game.c} que inicializa los 2 jugadores y el reloj de cada pirata.

\subsection{Paginado}
Habilitamos la paginaci\'on en nuestro sistema de la siguiente manera:
\begin{algorithmic}
    \State \tab ; Cargar directorio de paginas
    \State \tab ; Inicializar el directorio de paginas
    \State \tab call $mmu\_inicializar$
    \State \tab call $mmu\_inicializar\_dir\_kernel$
    \State \tab xor eax, eax
    \State \tab mov eax, 0x27000
    \State \tab mov cr3, eax
    \State \tab ; Habilitar paginacion
    \State \tab xor eax, eax
    \State \tab mov eax, cr0
    \State \tab or eax, 0x80000000
    \State \tab mov cr0, eax
\end{algorithmic}

Para mappear y unmapear definimos m\'etodos en el archivo \textit{mmu.c}, para ser mas preciso, mmu$\_$mapear$\_$pagina que devuelve void y recibe una direccion virtual y fisica para el que tiene que mapear, un cr3 que es nuestra dirección de comienzo del page$\_$directory$\_$address y flags de read-write y user-supervisor.
Durante el mapeo, lo que se hace es lo siguiente:
\begin{itemize}
    \item Se saca el índice del page directory shifteando la virtual para la derecha 22 bits.
    \item Se saca el índice del page table shifteando la virtual 10 bits para la izquierda y despues 22 para la derecha (así nos quedamos con lo del medio).
    \item Se busca si el page directory en ese índice está presente. Si no lo está se lo pone presente y se pide una pagina libre para el page table. Se establece como base$\_$address del page directory esta nueva página pedida shifteada 12 its para la derecha y en éste nuevo page table se pone como base$\_$address la fisica que nos pasan como parámetro shifteada 12 bits para la derecha.
    \item Si el page directory en ese índice sí esta presente, sólo se pide el page table al que apunta el base$\_$address de ese page directory. A éste último se le establece como base$\_$address la física que nos pasaron como parámetro shifteada 12 bits para la derecha.
    \item Luego sólo resta poner los bits correspondientes prendidos, apagados, o los read-write y user-supervisor que nos pasan como parámetro.
\end{itemize}

Finalmente, mara desmapear una pagina, usamos mmu$\_$unmapear$\_$pagina que toma una dirección virtual y un cr3. Lo que se hace para esto es:
\begin{itemize}
	\item Se shiftea 22 bits a la derecha la virtual pasada por parámetro para sacar el índice dle page directory.
	\item Se shiftea 10 bits para la izquierda y 22 para la derecha para quedarse con el índice del page table de la dirección virtual pasada por parámetro.
	\item Se va al page table de esa dirección virtual sacando lo que apunta la base$\_$address del page directory en el primer índice resuelto y se la pone como no presente.
\end{itemize}
Siempre que mapeamos o desmapeamos algo de la memoria llamamos a $tlbflush()$ para actualizar el caché de interrupciones llamado Translation Lookaside Buffer o TLB.

Durante el juego, debemos poder paginar de manera din\'amica las direcciones de los piratas que se agregan, para poder realizarlo utilizamos el siguiente m\'etodo:
\begin{algorithmic}
    \State \tab unsigned int $mmu\_proxima\_pagina\_fisica\_libre$() {
  \State \tab \tab unsigned int $pagina\_libre$ = $proxima\_pagina\_libre$;
  \State \tab \tab $proxima\_pagina\_libre$ += $PAGE\_SIZE$;
  \State \tab \tab return $pagina\_libre$;
\State \tab }
\end{algorithmic}

Los piratas (o tareas) en el juego van a tener que copiar su código a la página en donde están y luego, a medida que se vayan moviendo por el mapa, lo que se mueve es ese código por todo el mapa (que está representado desde la 0x800000 a 0x1520000 y 0x500000 a 0x121FFFF en la física).
Para ésto, en \textbf{mmu.c} contamos con las funciones memcpy y memmov. La primera usando el cr3 que esta en ejecución, mapea la página destino con ella misma (sólo se va a usar para escribir en memoria). Se copia los datos que se querían copiar a esa página y a continuación se desmapea. 
La función memmov es similar a la anterior, sólo que en vez de únicamente la página de destina se mapea la página de destina y la página de source, se traslada lo que esta source en destino y se desmapean ambas.

Hay otra función en \textbf{mmu.c} que es memcpyPila que lo único que hace es pasarle a la pila de la tarea un valor. Ésta función sólo se usa cuando se quieren pasar los parámetros de x e y a la tarea minero.

La última función importante de \textbf{mmu.c} es mmu$\_$inicializar$\_$dir$\_$pirata.
\newline
Ésta función primero mapea el kernel, ya que todas las tareas deben tener mapeados el kernel por si hay una interrupción de nivel 0 y se está ejecutando una tarea. Luego se mapean las posiciones donde empieza el pirata y sus 8 posiciones circundantes. Para esto tenemos la ayuda de dos funciones; pos2mapVir y pos2mapFis que dados un x y un y devuelve la dirección de memoria de la página de esas coordenadas. Ésto último, a diferencia del kernel, se mapea con nivel de usuario (es decir, 1).
 


\subsection{Manejo de Interrupciones}
Para poder atender los distintos tipos de interrupciones, definimos las tareas de atenci\'on en \textit{isr.asm}
Definimos una rutina para atender las interrupciones del reloj:
\begin{algorithmic}
    \State \tab global $\_isr32$
\State \tab $\_isr32$:
  \State \tab \tab; PRESERVAR REGISTROS
  \State \tab \tab pushad
  \State \tab \tab call $fin\_intr\_pic1$
  \State \tab \tab cmp byte [modoDebugPantalla], 1
  \State \tab \tab je .fin
\State \tab \tab  call $sched\_tick$
  \State \tab \tab str cx
  \State \tab \tab cmp ax, cx
  \State \tab \tab je .fin
  \State \tab \tab mov [$sched\_tarea\_selector$], ax
  \State \tab \tab jmp far [$sched\_tarea\_offset$]
\State \tab \tab .fin:
  \State \tab \tab ; RESTAURAR REGISTROS
  \State \tab \tab popad
  \State \tab \tab iret
\end{algorithmic}

Luego, para ateneder las interrupciones correspondientes al teclado lo hacemos de la siguiente manera:
\begin{algorithmic}
    \State \tab $global \_isr33$
\State \tab \tab $\_isr33$:
  \State \tab \tab pushad
  \State \tab \tab call $fin\_intr\_pic1$
  \State \tab \tab xor ax, ax
  \State \tab \tab in al, 0x60
  \State \tab \tab push eax
  \State \tab \tab call $game\_atender\_teclado$
  \State \tab \tab pop eax
  \State \tab \tab popad
  \State \tab \tab iret
\end{algorithmic}

El m\'etodo \textit{$game\_atender\_teclado$} esta definido en nuestro archivo \textit{game.c}. Este m\'etodo lo que hace, es imprimir en el rincon derecho superior de la pantalla la tecla que se presion\'o.
Cuenta con un switch, que evalua el caso de cada tecla posible, y en base a esto imprime lo que corresponde.

La interrupción del software la realizamos con
\begin{algorithmic}
\State global $\_$isr70
\State $\_$isr70:
\State \tab cmp eax, 3
\State \tab je .posicion
\State \tab pushad
\State \tab push ecx
\State \tab push eax
\State \tab call game$\_$syscall$\_$manejar
\State \tab call sched$\_$intercambiar$\_$por$\_$idle
\State \tab mov ax, 13$<<$3
\State \tab mov [sched$\_$tarea$\_$selector], ax
\State \tab jmp far [sched$\_$tarea$\_$offset]
\State \tab pop eax
\State \tab pop ecx
\State \tab jmp .fin
\State .posicion:
\State \tab mov edi, eax
\State \tab push ecx
\State \tab push edi
\State \tab call game$\_$syscall$\_$manejar
\State \tab pop ecx
\State \tab pop edi
\State \tab jmp .fin2
\State .fin:
\State \tab popad
\State .fin2:
\State \tab iret
\end{algorithmic}
Ésto compara si la interrupción es por la función posición, en cuyo caso se hace aparte para poder devolver el resultado en eax. En el otro caso, llamamos a manejar e intercambiamos por idle. Después hacemos el jmp far para cambiar la tarea.

Para atender las interrupciones de excepciones que describe el ejercicio 2 del TP, definimos una macro:
\begin{algorithmic}
\State \tab $\%$ macro ISR 1
\State \tab global $\_$isr$\%1$

\State \tab $\_$ isr$\%$1:
    \State \tab \tab mov eax, $\%1$
    \State \tab \tab imprimir$\_$texto$\_$mp desc$\_\%$1, desc$\_$len$\_\%$1, 0x07, 3, 0
    \State \tab \tab jmp $\$$

\State \tab %endmacro
\end{algorithmic}

De esta manera, podemos definir el mensaje correspondiente para cada interrupcion, y utilizando la macro, no tenemos que repetir el c\'odigo, ya que para todas las interrupciones va a trabajar de la misma manera. Va a imprimir en pantalla el nombre de la interrupci\'on que acaba de ocurrir.
Por ejemplo, cuando queremos identificar la interrupci\'on de "Divide Error", definimos el mensaje de la siguiente maner:

\begin{algorithmic}
\State \tab desc$\_$0 db   'Divide Error'
\State \tab desc$\_$len$\_$0 equ     $\$$ - desc$\_$0
\end{algorithmic}

De esta misma manera definimos todos los mensajes correspondientes a las interrupciones, en nuestro archivo \textit{isr.asm}.
\newline
En el kernel.asm las interrupciones están representadas por el siguiente código:
\begin{algorithmic}
\State call idt$\_$inicializar
\State lidt[IDT$\_$DESC]
\State call resetear$\_$pic
\State call habilitar$\_$pic
\end{algorithmic}

Donde lo que se hace es inicializar las entries de la idt, inclusive la entry de la syscall que necesita otro dpl (en este caso 3, de usuario) a diferencia del 0 que es kernel. Por eso hay IDT$\_$entry y IDT$\_$entry$\_$syscall en \textbf{idt.c}. La IDT se carga con la instrucción lidt con la dirección de la idt y se resetea y habilita el Programmable Interrupt Controller, que es el mismo al cual le avisamos en las isr's cuando se atiende cada una de las interrupciones. La linea $sti$ activa las interrupciones (set interrupts). Ésto lo hacemos después de cargar la tarea inicial.
\subsection{Tareas}

Completamos las TSS utilizando c\'odigo c, en el archivo \textit{tss.c}. Se necesita un idle que va a hacer como pivot en todo cambio de tareas y una inicial para que el primer context switch pueda tirar sus datos en algún lado, por eso no es importante qué tiene y se llena toda con ceros (o basura).

Debemos agregar los \textit{descriptores de TSS} correspondientes a la tarea \textit{idle} y la tarea \textit{inicial} a nuestra \textit{GDT}, para poder realizar esto, definimos un m\'etodo llamado \textit{tss$\_$agregar$\_$a$\_$gdt}

Luego necesitamos completar la TSS correspondiente a cada pirata.

En el kernel.asm que es lo que se corre en bochs, las tareas estan representadas por el siguiente código:
\begin{algorithmic}
\State call tss$\_$inicializar
\State call tss$\_$agregar$\_$a$\_$gdt
\end{algorithmic}
Esto lo que hace es inicializar los valores de las tss$\_$idle y tss$\_$inicial que están declaradas en \textbf{tss.h} y agregar sus respectivos descriptores a la GDT.
El código
\begin{algorithmic}
\State mov ax, 0x70
\State ltr ax
\end{algorithmic}
carga la tarea inicial y
\begin{algorithmic}
\State jmp 0x68:0
\end{algorithmic}
cambia a la tarea idle.
Ésta tarea tiene como cr3, base de la pila y stack el 0x27000, el eip de 0x16000 que es donde está el código de la tarea.
\newline
\newline
Cada pirata es una tarea distinta y tienen cada uno un cr3 distinto.
Cuando se lanza una tarea lo primero que se hace es agregar el descriptor de tss de ese pirata a la GDT y despues se completa la tss. Ésto se hace con los métodos tss$\_$agregar$\_$pirata$\_$a$\_$gdt y completarTssPirata, respectivamente. Ambos estan en el archivo \textbf{tss.c}
Las entradas de los descriptores de TSS en la GDT tienen como límite 0x0067 porque es el tamaño de la TSS. Se corren en nivel usuario, por eso su dpl es 3, el tipo según el manual de Intel tiene que ser 9 y la base apunta a la dirección de memoria de su correspondiente TSS. Para esto tenemos dos arreglos de TSS, uno para el jugador A y otro para el jugador B de size 8 (porque cada uno puede tener como máximo 8 piratas en ejecución). Por supuesto, todas estas entradas en la GDT se marcan como presente.
Cuando se completa una tss, esto es con el método completarTssPirata lo que se hace es lo siguiente:

\begin{itemize}

\item Pido una pagina de memoria libre que voy a utilizar para la pila de nivel cero de la tarea en cuestión. Le sumo 4KB (el tamaño de una página) para que el stack pointer apunte al tope de la pila.
\item Tomo el puntero a la tss a llenar.
\item Pongo como stack pointer de la pila de nivel 0 a la pagina descripta en el punto uno de ésta enumeración.
\item El stack segment de nivel kernel es el índice del segmento de datos de nivel kernel en la GDT shifteado 3 para sacar el table indicator y el RPL.
\item El eip va a ser el 0x400000 (requerimiento del TP, es la virtual a donde despues se va a copiar el código de la tarea).
\item Los eflags siempre que están activadas las interrupciones Intel dice que hay que ponerlos en 0x202.
\item El es, ss, ds, fs y gs todos van a tener el índice en la GDT del segmento de datos de usuario (shifteados también 3 para sacar el TI y el RPL).
\item El cs va a tener el índice de la gdt correspondiente a el segmento de código nivel user (shifteado 3 para sacar el TI y RPL).
\item Para todos estos segmentos hacemos un OR con 3 porque el DPL es 3 (user).
\item El manual de Intel indica que el iomap hay que ponerlo todo en 1.
\item El esp y ebp tendrán como valor el 401000, ya que según el enunciado del TP comparten el mismo espacio que la tarea y crecen desde la base. Se les resta 12 porque también en el enunciado se dice que las tareas esperan tener apilados sus 2 argumentos y una dirección de retorno y como estamos en 32 bits cada uno de esos 3 datos son 4 bytes.
\item El cr3 se usa la función mmu$\_$inicializar$\_$dir$\_$pirata que toma como parámetros al jugador de esa tarea y a la referencia de la tarea.
 
\end{itemize}

En el código \textbf{defines.h} hay una constante que es EMPIEZAN$\_$TSS que sirve para saber a partir de qué índice en la GDT, las entradas pertenecen a las tareas. En nuestro caso es 15.

\subsection{Scheduler}

Decidimos representar al \textit{Scheduler} de la siguiente manera:
\begin{algorithmic}
\State \tab int tareaActualA;
\State \tab int tareaActualB;
\State \tab uint proximaTareaA; //indice 0-7
\State \tab uint proximaTareaB; //indice 0-7
\State \tab uchar turnoPirata; //0 A, 1 B
\State \tab uchar turnoPirataActual; //0 A, 1 B
\State \tab uchar estaEnIdle; // 0 NO, 1 SI
\State \tab uint modoDebugPantalla;
\State \tab uint modoDebugActivado;
\State \tab int proxTareaAMuerta;
\State \tab int proxTareaBMuerta;
\end{algorithmic}

Para inicializar estos valores, definimos el siguiente m\'etodo:
\begin{algorithmic}
    \State \tab void sched$\_$inicializar$() \{$
    \State \tab \tab turnoPirata = 0;
    \State \tab \tab proximaTareaA = 0;
    \State \tab \tab estaEnIdle = 1;
    \State \tab \tab proximaTareaB = 0;
    \State \tab \tab modoDebugActivado = 1;
    \State \tab \tab tareaActualA = 0;
    \State \tab \tab tareaActualB = 0;
    \State \tab \tab turnoPirataActual = 0;
    \State \tab \tab proxTareaAMuerta = 0;
    \State \tab \tab proxTareaBMuerta = 0;
    \State \tab \tab modoDebugPantalla = 0; 
    \State \tab $\}$
\end{algorithmic}

El sistema de context-switch para las tareas es el denominado round-robin, es decir, le toca un ciclo de reloj a cada jugador.
El jugador que empieza el es A, la primera tarea que empieza a correr es la idle por eso está seteado. Ambos jugadores tienen como primer tarea a correr la 0 o sea la primera. El modoDebugActivado está seteado porque el enunciado dice que la primer excepción del procesador ya tiene que lanzar la pantalla y hay dos flags porque uno indica si está la pantalla debug y otra si está el debug activado.
proximaTareaA y proximaTareaB indican las proximas tarea a correr de cada jugador. Por último, la proxTareaAMuerta y proxTareaBMuerta indican cuales son los indices de los piratas de cada jugador que van a ser usados en caso de que se lance un pirata verde o azul, respectivamente.

Para atender a la proxima tarea, vamos a reutilizar la funci\'on que cambia las tareas de acuerdo al tick del clock, utilizamos a la funci\'on:
\begin{algorithmic}
    \State \tab unsigned int sched$\_$proxima$\_$a$\_$ejecutar$() \{$
    \State \tab \tab return sched$\_$tick$()$;
    \State \tab $\}$
\end{algorithmic}

Luego, para realizar lo correspondiente al tick del clock usamos sched$\_$tick, que a su vez llama a game$\_$tick que sólo actualiza todos los relojes, en concordancia a lo que pide el ejercicio


Por \'ultimo, otros m\'etodos que definimos en nuestro \textit{sched.c}
\begin{algorithmic}
    \State \tab void sched$\_$intercambiar$\_$por$\_$idle$() \{$
        \State \tab \tab estaEnIdle = 1;
    \State \tab $\}$
\end{algorithmic}

\begin{algorithmic}
    \State \tab void sched$\_$nointercambiar$\_$por$\_$idle$() \{$
        \State \tab \tab estaEnIdle = 0;
    \State \tab $\}$
\end{algorithmic}

\begin{algorithmic}
    \State \tab void sched$\_$toggle$\_$debug$() \{$
        \State \tab \tab if (modoDebugPantalla) $\{$
        \State \tab \tab \tab game$\_$modoDebug$\_$close$();$
        \State \tab \tab \tab modoDebugPantalla = FALSE;
        \State \tab \tab $\}$ else $\{$
        \State \tab \tab \tab if (modoDebugActivado) $\{$
        \State \tab \tab \tab \tab modoDebugActivado = TRUE;
        \State \tab \tab \tab $\}$ else $\{$
        \State \tab \tab \tab \tab modoDebugActivado = FALSE;
        \State \tab \tab	\tab $\}$
        \State \tab \tab $\}$
    \State \tab $\}$
\end{algorithmic}

Éste último método se llama cuando se aprieta la tecla $"y"$. Lo que hace es preguntar si está la pantalla del debug en el juego, en este caso la cierra y vuelve al juego, poniendo el flag en FALSE, que esta definido como cero. Si no está la pantalla cheque si ya se apretó la tecla $"y"$ y el modoDebugActivado está activado o no, haciendo un simple toggle de éste último valor.
\newline
A su vez, el scheduler se tiene que encargar de sacar una tarea cuando ésta muere y agregar una tarea cuando se lanza. Ésto se hace con sched$\_$sacar y sched$\_$agregar, respectivamente. Lo que hace el primero es fijarse cual es la próxima tarea muerta del jugador en cuestión y actualiza si la tarea que se murió viene antes. El sched$\_$agregar se fija cual esta muerto de las tareas del jugador y actualiza su variable próxima tarea muerta.
\newline
El scheduler si inicialize en el kernel.asm cuando se hace
\begin{algorithmic}
\State call sched$\_$inicializar
\end{algorithmic}


\subsection{modoDebug}
Para el modo debug usamos 2 flags que están en la estructura del scheduler, modoDebugActivado y modoDebugPantalla.

\begin{algorithmic}
	\State \tab pushad
    \State \tab call game$\_$pirata$\_$explotoisr
    \State \tab cmp byte [modoDebugActivado], 1
    \State \tab jne .fin
    
    \State \tab mov eax, cr0
    \State \tab push eax
    \State \tab mov eax, cr2
    \State \tab push eax
    \State \tab mov eax, cr3
    \State \tab push eax
    \State \tab mov eax, cr4
    \State \tab push eax

    \State \tab push cs ; push segmentos
    \State \tab push ds
    \State \tab push es
    \State \tab push fs
    \State \tab push gs
    \State \tab push ss

    \State \tab push esp ; push array con toda la info de los registros
    \State \tab call game$\_$modoDebug$\_$open
  	\State .fin:
    \State \tab jmp 0x68:0 ;jumpeo a la idle\end{algorithmic}
    
Cuando se produce una excepción del procesador se va a la rutina de excepción de interrupciones de arriba.
Ésta rutina lo que hace es:
\begin{itemize}
\item Pushea los registros de proposito general.
\item Llama a game$\_$pirata$\_$explotoisr que mata a la tarea en cuestión, es decir la saca del scheduler, la borra de la pantalla, reestablece el clock del pirata en cero, etc.
\item Pushea los registros de control a la pila.
\item Pushea los segmentos a la pila.
\item Pushea el esp, que es lo que se le pasa como parámetro a la función game$\_$modoDebug$\_$open.
\item Si el flag modoDebugActivado está apagado no se hace nada de ésto y simplemente se va a la tarea Idle, haciendo un jmp far a su selector de segmento en la GDT. 
\end{itemize}

Las funciones game$\_$modoDebug$\_$open y su contraparte game$\_$modoDebug$\_$close están ambas definidas en el archivo \textbf{game.c}. La primera toma el puntero pasado por parámetro para ir para atras en la pila y guardar todos los registros y valores necesarios para la pantalla debug. También guarda el estado de la pantalla del juego en el buffer que denominamos como mapa$\_$backup  con un puntero al egmento de video antes de popear la pantalla debug y pone el flag modoPantallaDebug en 1.
La segunda sólo reestablece la pantalla del juego utilizando el puntero al segmento de video y el buffer mapa$\_$backup.
\newline
En el archivo \textbf{defines.h} hay una serie de constantes que nos sirven a la hora de implementar el modoDebug. Éstas son:
\begin{itemize}
\item DEBUG$\_$REGISTROS$\_$X: La coordenada X del borde superior izquierdo del rectángulo imaginario donde se empiezan a listar los datos.
\item DEBUG$\_$REGISTROS$\_$Y: La coordenada Y del borde superior izquierdo del rectángulo imaginario donde se empiezan a listar los datos.
\item DEBUG$\_$CORNER$\_$X: La coordenada X del borde superior izquierdo de la pantalla debug.
\item DEBUG$\_$CORNER$\_$Y: La coordenada Y del borde superior izquierdo de la pantalla debug. 
\item DEBUG$\_$WIDTH: El ancho de la pantalla debug.
\item DEBUG$\_$HEIGHT: El alto de la pantalla debug.
\end{itemize}

Por último, por afuera del modoDebug, se implementó un nuevo syscall, con un tipo 4 para que sea diferente a los tipos de los syscalls provistos por la cátedra, que imprime en pantalla el valor que se le pasa. Éste syscall resultó muy útil cuando se quería inspeccionar algún dato o valor desde los códigos de las tareas. Es el último método definido en el archivo \textbf{syscall.h} 

