Cuando encendemos nuestra computadora, la misma inicia en \textit{Modo Real}. En este modo tenemos una memoria disponible de 1mb, no disponemos de privilegios de ningun tipo, y debemos alinear la memoria a 16 bits para poder operar.
Para manejar los distintos tipos de interrupciones, disponemos de rutinas de atencion a las mismas. Podemos acceder a instrucciones de cualquier tipo.
Estando en Modo real, lo unico que vamos a hacer, es preparar el sistema para poder pasar a \textit{Modo Protegido}, y realizar todo lo que necesitamos que haga nuestro sistema.
Para nuestro sistema, definimos el c\'odigo de \textit{Modo Real} en el archivo \textit{kernel.asm}:

\begin{algorithmic}
\State \tab BITS 16
\State \tab start:
    \State \tab \tab ; Deshabilitar interrupciones
    \State \tab \tab cli
\newline
\State \tab \tab     ; Cambiar modo de video a 80 X 50
    \State \tab \tab mov ax, 0003h
    \State \tab \tab int 10h ; set mode 03h
    \State \tab \tab xor bx, bx
    \State \tab \tab mov ax, 1112h
    \State \tab \tab int 10h ; load 8x8 font
\newline
    \State \tab \tab ; Imprimir mensaje de bienvenida
    \State \tab \tab $imprimir\_texto\_mr \ iniciando\_mr\_msg, iniciando\_mr\_len$, 0x07, 0, 0
    \State \tab \tab ; Habilitar A20
    \State \tab \tab call $habilitar\_A20$
    \State \tab \tab ; Cargar la GDT
    \State \tab \tab lgdt $[GDT\_DESC]$
    \newline
    \State \tab \tab ; Setear el bit PE del registro CR0
    \State \tab \tab mov eax, cr0
    \State \tab \tab or eax, 1
    \State \tab \tab mov cr0, eax
    \newline
    \State \tab \tab ; Saltar a modo protegido
    \State \tab \tab jmp 0x58:mp

    \end{algorithmic}

Mediante este set de instrucciones estamos preparando nuestro sistema para poder pasar a \textit{Modo Protegido}.
Dentro del codigo, estamos habilitando A20 que nos da accesos a direcciones superiores a los 2$^{20}$ bits.\newline
Cargamos la \textit{GDT} (Global Descriptor Table).
La \textit{GDT} es una tabla ubicada en memoria que define los descriptores de segmentos de memoria en principio.
Luego seteamos el bit \textit{PE} del registro \textit{CR0}.
\newline
Una vez activado, vamos a saltar a \textit{Modo Protegido}. Este salto lo realizamos utilizando un far jump al descriptor de segmento de c\'odigo con nivel de privilegio de sistema. Es el jmp que realizamos en la \'ultima linea.
